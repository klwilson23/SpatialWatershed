\documentclass[]{article}
\usepackage{lmodern}
\usepackage{amssymb,amsmath}
\usepackage{ifxetex,ifluatex}
\usepackage{fixltx2e} % provides \textsubscript
\ifnum 0\ifxetex 1\fi\ifluatex 1\fi=0 % if pdftex
  \usepackage[T1]{fontenc}
  \usepackage[utf8]{inputenc}
\else % if luatex or xelatex
  \ifxetex
    \usepackage{mathspec}
  \else
    \usepackage{fontspec}
  \fi
  \defaultfontfeatures{Ligatures=TeX,Scale=MatchLowercase}
\fi
% use upquote if available, for straight quotes in verbatim environments
\IfFileExists{upquote.sty}{\usepackage{upquote}}{}
% use microtype if available
\IfFileExists{microtype.sty}{%
\usepackage{microtype}
\UseMicrotypeSet[protrusion]{basicmath} % disable protrusion for tt fonts
}{}
\usepackage[margin=1in]{geometry}
\usepackage{hyperref}
\hypersetup{unicode=true,
            pdftitle={Managing for ecological surprises in metapopulations},
            pdfborder={0 0 0},
            breaklinks=true}
\urlstyle{same}  % don't use monospace font for urls
\usepackage{color}
\usepackage{fancyvrb}
\newcommand{\VerbBar}{|}
\newcommand{\VERB}{\Verb[commandchars=\\\{\}]}
\DefineVerbatimEnvironment{Highlighting}{Verbatim}{commandchars=\\\{\}}
% Add ',fontsize=\small' for more characters per line
\usepackage{framed}
\definecolor{shadecolor}{RGB}{248,248,248}
\newenvironment{Shaded}{\begin{snugshade}}{\end{snugshade}}
\newcommand{\KeywordTok}[1]{\textcolor[rgb]{0.13,0.29,0.53}{\textbf{#1}}}
\newcommand{\DataTypeTok}[1]{\textcolor[rgb]{0.13,0.29,0.53}{#1}}
\newcommand{\DecValTok}[1]{\textcolor[rgb]{0.00,0.00,0.81}{#1}}
\newcommand{\BaseNTok}[1]{\textcolor[rgb]{0.00,0.00,0.81}{#1}}
\newcommand{\FloatTok}[1]{\textcolor[rgb]{0.00,0.00,0.81}{#1}}
\newcommand{\ConstantTok}[1]{\textcolor[rgb]{0.00,0.00,0.00}{#1}}
\newcommand{\CharTok}[1]{\textcolor[rgb]{0.31,0.60,0.02}{#1}}
\newcommand{\SpecialCharTok}[1]{\textcolor[rgb]{0.00,0.00,0.00}{#1}}
\newcommand{\StringTok}[1]{\textcolor[rgb]{0.31,0.60,0.02}{#1}}
\newcommand{\VerbatimStringTok}[1]{\textcolor[rgb]{0.31,0.60,0.02}{#1}}
\newcommand{\SpecialStringTok}[1]{\textcolor[rgb]{0.31,0.60,0.02}{#1}}
\newcommand{\ImportTok}[1]{#1}
\newcommand{\CommentTok}[1]{\textcolor[rgb]{0.56,0.35,0.01}{\textit{#1}}}
\newcommand{\DocumentationTok}[1]{\textcolor[rgb]{0.56,0.35,0.01}{\textbf{\textit{#1}}}}
\newcommand{\AnnotationTok}[1]{\textcolor[rgb]{0.56,0.35,0.01}{\textbf{\textit{#1}}}}
\newcommand{\CommentVarTok}[1]{\textcolor[rgb]{0.56,0.35,0.01}{\textbf{\textit{#1}}}}
\newcommand{\OtherTok}[1]{\textcolor[rgb]{0.56,0.35,0.01}{#1}}
\newcommand{\FunctionTok}[1]{\textcolor[rgb]{0.00,0.00,0.00}{#1}}
\newcommand{\VariableTok}[1]{\textcolor[rgb]{0.00,0.00,0.00}{#1}}
\newcommand{\ControlFlowTok}[1]{\textcolor[rgb]{0.13,0.29,0.53}{\textbf{#1}}}
\newcommand{\OperatorTok}[1]{\textcolor[rgb]{0.81,0.36,0.00}{\textbf{#1}}}
\newcommand{\BuiltInTok}[1]{#1}
\newcommand{\ExtensionTok}[1]{#1}
\newcommand{\PreprocessorTok}[1]{\textcolor[rgb]{0.56,0.35,0.01}{\textit{#1}}}
\newcommand{\AttributeTok}[1]{\textcolor[rgb]{0.77,0.63,0.00}{#1}}
\newcommand{\RegionMarkerTok}[1]{#1}
\newcommand{\InformationTok}[1]{\textcolor[rgb]{0.56,0.35,0.01}{\textbf{\textit{#1}}}}
\newcommand{\WarningTok}[1]{\textcolor[rgb]{0.56,0.35,0.01}{\textbf{\textit{#1}}}}
\newcommand{\AlertTok}[1]{\textcolor[rgb]{0.94,0.16,0.16}{#1}}
\newcommand{\ErrorTok}[1]{\textcolor[rgb]{0.64,0.00,0.00}{\textbf{#1}}}
\newcommand{\NormalTok}[1]{#1}
\usepackage{graphicx,grffile}
\makeatletter
\def\maxwidth{\ifdim\Gin@nat@width>\linewidth\linewidth\else\Gin@nat@width\fi}
\def\maxheight{\ifdim\Gin@nat@height>\textheight\textheight\else\Gin@nat@height\fi}
\makeatother
% Scale images if necessary, so that they will not overflow the page
% margins by default, and it is still possible to overwrite the defaults
% using explicit options in \includegraphics[width, height, ...]{}
\setkeys{Gin}{width=\maxwidth,height=\maxheight,keepaspectratio}
\IfFileExists{parskip.sty}{%
\usepackage{parskip}
}{% else
\setlength{\parindent}{0pt}
\setlength{\parskip}{6pt plus 2pt minus 1pt}
}
\setlength{\emergencystretch}{3em}  % prevent overfull lines
\providecommand{\tightlist}{%
  \setlength{\itemsep}{0pt}\setlength{\parskip}{0pt}}
\setcounter{secnumdepth}{0}
% Redefines (sub)paragraphs to behave more like sections
\ifx\paragraph\undefined\else
\let\oldparagraph\paragraph
\renewcommand{\paragraph}[1]{\oldparagraph{#1}\mbox{}}
\fi
\ifx\subparagraph\undefined\else
\let\oldsubparagraph\subparagraph
\renewcommand{\subparagraph}[1]{\oldsubparagraph{#1}\mbox{}}
\fi

%%% Use protect on footnotes to avoid problems with footnotes in titles
\let\rmarkdownfootnote\footnote%
\def\footnote{\protect\rmarkdownfootnote}

%%% Change title format to be more compact
\usepackage{titling}

% Create subtitle command for use in maketitle
\newcommand{\subtitle}[1]{
  \posttitle{
    \begin{center}\large#1\end{center}
    }
}

\setlength{\droptitle}{-2em}

  \title{Managing for ecological surprises in metapopulations}
    \pretitle{\vspace{\droptitle}\centering\huge}
  \posttitle{\par}
  \subtitle{Supplemental materials}
  \author{Kyle Logan Wilson\(^1\), Colin Bailey\(^1\), William Atlas\(^1\), and
Doug Braun\(^2\)\\[2\baselineskip]\(^1\)Earth to Ocean Research Group,
Simon Fraser University\\
\(^2\)Fisheries \& Oceans Canada}
    \preauthor{\centering\large\emph}
  \postauthor{\par}
      \predate{\centering\large\emph}
  \postdate{\par}
    \date{02 May 2019}

\usepackage{wrapfig}
\usepackage{lipsum}
\usepackage{float}

\begin{document}
\maketitle

\subsection{Metapopulation model}\label{metapopulation-model}

\subsubsection{Local \& metapopulation
dynamics}\label{local-metapopulation-dynamics}

Our metapopulation is defined by a set of local populations \(N_p\) with
time-dynamics that follows birth (i.e., recruitment \emph{R}),
immigration, death, and emigration (BIDE) processes:

\(N_{it}= R_{it}\epsilon_{it}+I_{it}-D_{it}-E_{it}\)

where \(N_{it+1}\) is the number of adults in patch \emph{i} at time
\emph{t}, \(R_{it}\) is number of recruits, \(I_{it}\) is number of
recruits immigrating into patch \emph{i} from any other patch,
\(D_{it}\) is number of recruits that die due to disturbance regime,
\(E_{it}\) is the number of recruits emigrating from patch \emph{i} into
any other patch, and \(\epsilon_{it}\) is stochasticity in recruitment.

Resoure monitoring often occurs at the scale of the metapopulation,
hence we define metapopulation adults as:

\({MN}_t = \sum_{i=1}^{N_p} N_{it}\)

with metapopulation recruits:

\(MR_t = \sum_{i=1}^{N_p} R_{it}\)

Local patch recruitment at time \emph{t} depended on adult densities at
\emph{t-1} and followed a reparameterized Beverton-Holt function:

\(R_{it}=\cfrac{\alpha_iN_{it-1}}{1+\cfrac{\alpha_i-1}{\beta_i}N_{it-1}}\)

where \(\alpha_i\) is the recruitment compensation ratio and \(\beta_i\)
is local patch carrying capacity.

For example, in a two patch model that varies \(\alpha_i\) and
\(\beta_i\) parameters such that

\begin{Shaded}
\begin{Highlighting}[]
\NormalTok{alpha <-}\StringTok{ }\KeywordTok{c}\NormalTok{(}\DecValTok{2}\NormalTok{, }\DecValTok{4}\NormalTok{)}
\NormalTok{beta <-}\StringTok{ }\KeywordTok{c}\NormalTok{(}\DecValTok{100}\NormalTok{, }\DecValTok{200}\NormalTok{)}
\end{Highlighting}
\end{Shaded}

Management often monitors metapopulation resources as the aggregate of
all local populations. In this way, recruitment compensation from local
patches \(\alpha_i\) gets averaged across the metapopulation leading
mean compensation \(\bar{\alpha}\) of 3. Likewise, the total carrying
capacity of the metapopulation \(\bar{\beta}\) becomes the summation of
local patch carrying capacities \(\sum\beta_i\), which is 300. This
scale of monitoring generates the following local patch and
metapopulation dynamics:

\begin{figure}[H]

{\centering \includegraphics{Managing_for_ecological_surprises_in_metapopulations_makeHTML_files/figure-latex/recruit curves-1} 

}

\caption{Metapopulation and local patch recruitment dynamics.}\label{fig:recruit curves}
\end{figure}

\subsubsection{Creating the spatial
networks}\label{creating-the-spatial-networks}

The next aspect to our metapopulation model is connecting the set of
patches to one another. We need to specify the number of patches, their
arrangements (i.e., connections), and how far apart they are from one
another. We followed some classic metapopulation and source-sink
arrangements to create four networks that generalize across a few
real-world topologies: a linear habitat network (e.g., coastline), a
dendritic or branching network (e.g., coastal rivers), a star network
(e.g., mountain \& valley), and a complex network (e.g., terrestrial
plants).

To make networks comparable, each spatial network type needs the same
leading parameters (e.g., \(N_p\) and \(\bar{d}\)) . In this case for
number of patches, we set \(N_p\) to \texttt{16} and \(\bar{d}\) to
\texttt{1} unit (distance units are arbitrary). We used the
\texttt{igraph} package and some custom code to arrange our spatial
networks as the following:

\begin{figure}[H]

{\centering \includegraphics{Managing_for_ecological_surprises_in_metapopulations_makeHTML_files/figure-latex/networks-1} 

}

\caption{Four spatial network topologies.}\label{fig:networks}
\end{figure}

Note that distances between neighbor patches in the above networks are
equal.

An example dispersal matrix for the complex network:

\begin{verbatim}
##   A B E F C G D H I J K L M N O P
## A 0 1 1 1 2 2 3 3 2 2 2 3 3 3 3 3
## B 1 0 1 1 1 1 2 2 2 2 2 2 3 3 3 3
## E 1 1 0 1 2 2 3 3 1 1 2 3 2 2 2 3
## F 1 1 1 0 1 1 2 2 1 1 1 2 2 2 2 2
## C 2 1 2 1 0 1 1 1 2 2 2 2 3 3 3 3
## G 2 1 2 1 1 0 1 1 2 1 1 1 2 2 2 2
## D 3 2 3 2 1 1 0 1 3 2 2 2 3 3 3 3
## H 3 2 3 2 1 1 1 0 3 2 1 1 3 2 2 2
## I 2 2 1 1 2 2 3 3 0 1 2 3 1 1 2 3
## J 2 2 1 1 2 1 2 2 1 0 1 2 1 1 1 2
## K 2 2 2 1 2 1 2 1 2 1 0 1 2 1 1 1
## L 3 2 3 2 2 1 2 1 3 2 1 0 3 2 1 1
## M 3 3 2 2 3 2 3 3 1 1 2 3 0 1 2 3
## N 3 3 2 2 3 2 3 2 1 1 1 2 1 0 1 2
## O 3 3 2 2 3 2 3 2 2 1 1 1 2 1 0 1
## P 3 3 3 2 3 2 3 2 3 2 1 1 3 2 1 0
\end{verbatim}

\subsubsection{Dispersal}\label{dispersal}

Dispersal from patch \emph{i} into patch \emph{j} depends on constant
dispersal rate \(\omega\) (defined as the proportion of total local
recruits that will disperse) and an exponential distance-decay function
between \emph{i} and \emph{j} with distance cost to dispersal \(m\)
following:

\(E_{ij(t)}=\omega R_{it}p_{ij}\)

where \(E_{ij}\) is the total dispersing animals from patch \emph{i}
into patch \emph{j} and probability of dispersal between patches
\(p_{ij}\):

\(p_{ij}=\dfrac{e^{-md_{ij}}}{\sum\limits_{\substack{j=1 \\ j\neq i}}^{N_p} e^{-md_{ij}}}\)

where \(d_{ij}\) is the pairwise distance between patches. The summation
term in the denominator normalizes the probability of moving to any
patch to between 0 and 1. With \(\bar{d}= 1\), \(m=0.5\),
\(\omega=0.1\), \(R_{it}=100\) in a linear network:

\begin{figure}[H]

{\centering \includegraphics{Managing_for_ecological_surprises_in_metapopulations_makeHTML_files/figure-latex/dispersal-1} 

}

\caption{Example dispersal patterns across linear network.}\label{fig:dispersal}
\end{figure}

\subsubsection{Recruitment
stochasticity}\label{recruitment-stochasticity}

Enter some text here

\paragraph{Spatio-temporal
correlations}\label{spatio-temporal-correlations}

Enter some more text here

\subsubsection{Disturbance regimes}\label{disturbance-regimes}

In all scenarios, disturbance was applied after \texttt{50} years of
equilibrating the metapopulaton at pristine conditions. At year
\texttt{50+1}, we applied the disturbance regime (the regime varied from
\emph{uniform}, \emph{localized, random}, \emph{localized, extirpation},
and \emph{localized, targeted} - see \emph{Scenarios} below).
Disturbance immediately removes a set proportion of the metapopulation
adults at that time (i.e., \texttt{0.9} of \(MN_{t=51}\)). Once applied,
the metapopulation is no longer disturbed and spatio-temporal dynamics
emerge naturally from these new conditions.

\subsection{Emergent outcomes}\label{emergent-outcomes}

We measured the following ecological outcomes that have direction
application for conservation metrics.

\begin{enumerate}
\def\labelenumi{\arabic{enumi}.}
\item
  Recovered \& recovery rate after disturbance - the number of
  simulations where metapopulation abundance averaged \texttt{1.0}
  carrying capacity for \texttt{5} consecutive years post-disturbance
  and the number of years it took to get there.
\item
  Extinction \& extinction rate - the number of simulations where
  metapopulation abundance was \texttt{\textless{}0.05} carrying
  capacity and the mean number of years it took to get there.
\item
  Patch occupancy - the mean number of patches with
  \texttt{\textgreater{}0.05} local carrying capacity.
\item
  Spatial or temporal variance
\item
  Temporal variation in source, sink, pseudo-sink defined as:

  \begin{enumerate}
  \def\labelenumii{\alph{enumii}.}
  \tightlist
  \item
    Sources provide surplus recruits and net emigrants such that:
    \((R_{it}>N_{it})\) \& \((E_{it}>I_{it})\)
  \item
    Sinks consume recruits and net immigrants such that:
    \((R_{it} < N_{it})\) \& \((E_{it} < I_{it})\)
  \item
    Pseudo-sinks would provide surplus recruits in the absence of
    dispersal such that \((R_{it}>N_{it})\) but
    \((R_{it}+E_{it}) < (R_{it-1}+I_{it-1}-E_{it-1})\)
  \end{enumerate}
\item
  Fit stock-recruitment model to aggregate of metapopulation to
  estimate:

  \begin{enumerate}
  \def\labelenumii{\alph{enumii}.}
  \tightlist
  \item
    Relative bias in recruitment compensaton ratio compared to true
    metapopulation average
  \item
    Relative bias in metapopulation carrying capacities compared to true
    sum of carrying capacities across metapopulation
  \item
    Relative bias in expected recruitment production to true recruitment
    production across all patches
  \end{enumerate}
\end{enumerate}

\subsubsection{Monitoring \& management at
aggregate-scale}\label{monitoring-management-at-aggregate-scale}

While true metapopulation dynamics are controlled by local patch
dynamics and dispersal such that:

\(N_{it}= R_{it}\epsilon_{it}+I_{it}-D_{it}-E_{it}\)

\(R_{it}=\cfrac{\alpha_iN_{it-1}}{1+\cfrac{\alpha_i-1}{\beta_i}N_{it-1}}\)

\({MN}_t = \sum_{i=1}^{N_p} N_{it}\)

\({MR}_t = \sum_{i=1}^{N_p} R_{it}\)

natural resoure managers often monitors and manages at the scale of the
metapopulation. Hence, management at this scale inherently defines the
stock-recruitment dynamics of the aggregate complex of patches (i.e.,
metapopulation) as:

\({MR}_{t}=\cfrac{\hat{\alpha_t}{MN}_{t-1}}{1+\cfrac{\hat{\alpha_t}-1}{\hat{\beta_t}}{MN}_{it-1}}\)

where \(\hat{\alpha_t}\) is the estimated compensation ratio averaged
across the metapopulation at time \emph{t} and \(\hat{\beta_t}\) is the
estimated carrying capacity of the entire metapopulation. Necessarily,
these estimates emerge from monitoring data collected across all patches
and are sensitive to the quality of the data and how local patches
perform through time. For example, temporal shifts in productivity
regimes may be masked if most of the data were sampled before the regime
shift. To help surmount these issues, modern resource assessments use
data weighting and penalties (i.e., priors) when fitting models to data.

In our assessment, we weighted recent years of sampling over more
distant years such that:

\begin{figure}[H]

{\centering \includegraphics{Managing_for_ecological_surprises_in_metapopulations_makeHTML_files/figure-latex/unnamed-chunk-1-1} 

}

\caption{Likelihood weighting for samples collected over time from current year of sampling.}\label{fig:unnamed-chunk-1}
\end{figure}

Furthermore, we used penalized normal likelihoods on both
\(\hat{\alpha_t}\) and \(\hat{\beta_t}\) such that:

\(\hat{\alpha_t} \sim N(\mu=\hat{\alpha_{t-1}},\sigma=3\hat{\alpha_{t-1}})\)

and

\(\hat{\beta_t} \sim N(\mu=\hat{\beta_{t-1}},\sigma=3\hat{\beta_{t-1}})\)

where \(\mu=\hat{\alpha_{t-1}}\) and \(\mu=\hat{\beta_{t-1}}\)
represents the best estimates from the previous assessment and the
\texttt{3} in the \(\sigma\) term represents a 300\% coefficient of
variation. We used these penalized likelihoods to fit the above
aggregate stock-recruitment model with \emph{lognormal} error to the
metapopulation stock-recruit data collected at time \emph{t}. We used
the following function and fitted to the below \(\theta\) parameters
(termed \texttt{theta} in the function \texttt{optim()} using the
\texttt{L-BFGS-B} optimizer with a lower bound on \(\hat{\alpha}\) of
\texttt{1.01} (i.e., constrained to be at least above replacement).

\begin{Shaded}
\begin{Highlighting}[]
\NormalTok{SRfn <-}\StringTok{ }\ControlFlowTok{function}\NormalTok{(theta) \{}
\NormalTok{    a.hat <-}\StringTok{ }\NormalTok{theta[}\DecValTok{1}\NormalTok{]}
\NormalTok{    b.hat <-}\StringTok{ }\KeywordTok{exp}\NormalTok{(theta[}\DecValTok{2}\NormalTok{])}
\NormalTok{    sd.hat <-}\StringTok{ }\KeywordTok{exp}\NormalTok{(theta[}\DecValTok{3}\NormalTok{])}
\NormalTok{    rec.mean <-}\StringTok{ }\NormalTok{(a.hat }\OperatorTok{*}\StringTok{ }\NormalTok{spawnRec}\OperatorTok{$}\NormalTok{spawners)}\OperatorTok{/}\NormalTok{(}\DecValTok{1} \OperatorTok{+}\StringTok{ }\NormalTok{((a.hat }\OperatorTok{-}\StringTok{ }\DecValTok{1}\NormalTok{)}\OperatorTok{/}\NormalTok{b.hat) }\OperatorTok{*}\StringTok{ }\NormalTok{spawnRec}\OperatorTok{$}\NormalTok{spawners)}
    \CommentTok{# negative log likelihood on recruitment parameters}
\NormalTok{    nll <-}\StringTok{ }\OperatorTok{-}\DecValTok{1} \OperatorTok{*}\StringTok{ }\KeywordTok{sum}\NormalTok{(}\KeywordTok{dlnorm}\NormalTok{(spawnRec}\OperatorTok{$}\NormalTok{recruits, }\DataTypeTok{meanlog =} \KeywordTok{log}\NormalTok{(rec.mean), }\DataTypeTok{sdlog =}\NormalTok{ sd.hat, }
        \DataTypeTok{log =} \OtherTok{TRUE}\NormalTok{) }\OperatorTok{*}\StringTok{ }\NormalTok{spawnRec}\OperatorTok{$}\NormalTok{weights, }\DataTypeTok{na.rm =} \OtherTok{TRUE}\NormalTok{)}
    \CommentTok{# penalized likelihood on estimated alpha}
\NormalTok{    penalty1 <-}\StringTok{ }\OperatorTok{-}\KeywordTok{dnorm}\NormalTok{(a.hat, alphaLstYr, }\DecValTok{3} \OperatorTok{*}\StringTok{ }\NormalTok{alphaLstYr, }\DataTypeTok{log =} \OtherTok{TRUE}\NormalTok{)}
    \CommentTok{# penalized likelihood on estimated carrying capacity}
\NormalTok{    penalty2 <-}\StringTok{ }\OperatorTok{-}\KeywordTok{dnorm}\NormalTok{(b.hat, metaKLstYr, }\DecValTok{3} \OperatorTok{*}\StringTok{ }\NormalTok{metaKLstYr, }\DataTypeTok{log =} \OtherTok{TRUE}\NormalTok{)}
\NormalTok{    jnll <-}\StringTok{ }\KeywordTok{sum}\NormalTok{(}\KeywordTok{c}\NormalTok{(nll, penalty1, penalty2), }\DataTypeTok{na.rm =} \OtherTok{TRUE}\NormalTok{)}
    \KeywordTok{return}\NormalTok{(jnll)}
\NormalTok{\}}
\end{Highlighting}
\end{Shaded}

This above function allows us to assess the bias in \(\hat{\alpha_t}\),
\(\hat{\beta_t}\), and \(\hat{MR_t}\) compared to true \(\bar{\alpha}\),
\(\bar{\beta}\), and \({MR}_t\) across the metapopulation. This then
allows us to see how much information management \& monitoring programs
are missing when they assess metapopulations at the aggregate (rather
than local) scales.

\subsection{Scenarios}\label{scenarios}

We tested all combinations of the following eight proccesses (below) and
ran a \texttt{100} bootstraps per scenario to estimate the mean for each
of the above outcomes.

\begin{enumerate}
\def\labelenumi{\arabic{enumi}.}
\item
  Homogenous and spatially variable recruitment compensation ratio
  across patches, i.e.~intrinsic rate of population growth
  (\(\alpha_i\)).
\item
  Homogenous and spatially variable local carrying capacity across
  patches, i.e.~asymptote of expected recruits at high adult densities
  (\(\beta_i\))
\item
  Disturbances where a proportion of individuals removed from
  metapopulation (e.g., \texttt{0.90}) occurs.

  \begin{enumerate}
  \def\labelenumii{\alph{enumii}.}
  \tightlist
  \item
    \emph{uniform} - random individuals removed at equal vulnerability
    across all patches.
  \item
    \emph{localized, random} - random individuals removed from randomly
    selected subset of patches (as long as target loss can be achieved
    in subset)
  \item
    \emph{localized, extirpation} - total extirpation of randomly
    selected subset of patches (as long as target loss can be achieved
    in subset)
  \item
    \emph{localized, targeted} random individuals removed from subset of
    patches selected based on productivity (i.e., most productive
    patches are targeted so long as target loss can be achieved in
    subset).
  \end{enumerate}
\item
  Density-independent dispersal rates \(\omega\) from 0 to 20\% of
  individuals within a patch will disperse.
\item
  Topology of the spatial networks with linear, dendritic, star, and
  complex networks. Each network with \(N_p\) of \texttt{16} and
  distance between patches \(\bar{d}\) of \texttt{1}.
\item
  Stochastic recruitment deviates from low, medium, high coefficient of
  variation on lognormal error. Generate stochasticity in time-dynamics
  via random recruitement deviates away from expected.
\item
  Temporal correlation in recruitment deviates from low, medium, high
  correlation (i.e., good year at time \emph{t} begets good year at time
  \emph{t+1}).
\item
  Spatial correlation in recruitment deviates among patches from low,
  medium, to high correlation (i.e., neighboring patches go up or down
  together).
\end{enumerate}

\subsubsection{Example results}\label{example-results}

\begin{figure}[H]

{\centering \includegraphics{Managing_for_ecological_surprises_in_metapopulations_makeHTML_files/figure-latex/example results-1} 

}

\caption{Spatial recovery regime of metapopulation with linear topology through time (top left) and space (top right). Recruitment dynamics before and 10 years after disturbance (bottom left). Relative bias in aggregate-scale estimates of carrying capacity, compensation ratio, and recruitment production in recovery phase (bottom right).}\label{fig:example results}
\end{figure}

\subsection{Simulation test \&
bootstrap}\label{simulation-test-bootstrap}

\subsubsection{General patterns}\label{general-patterns}


\end{document}
